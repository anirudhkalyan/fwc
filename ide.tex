\documentclass{article}

% Language setting
% Replace `english' with e.g. `spanish' to change the document language
\usepackage[english]{babel}

% Set page size and margins
% Replace `letterpaper' with `a4paper' for UK/EU standard size
\usepackage[letterpaper,top=2cm,bottom=2cm,left=3cm,right=3cm,marginparwidth=1.75cm]{geometry}

% Useful packages
\usepackage{multicol}
\usepackage{amsmath}
\usepackage{graphicx}
\usepackage{watermark}
\usepackage{array}
\usepackage{blindtext}
\usepackage{hyperref}
\usepackage{tabularx}
\usepackage[utf8]{inputenc}
\thiswatermark{\centering
\put(330,-150){\includegraphics[scale=02]{IIT.png}}}
\title{P.O.S Canonical Form For Truth Table}
\author{Kalyan}

\begin{document}
\maketitle
\begin{multicols}{2}
\tableofcontents

\begin{abstract}
 This manual shows how to use Arduino with 7447 and sevensegment dispaly to represent pos canonical form for function 'F' in truth table.
   
\begin{tabularx}{0.4\textwidth} { 
  | >{\centering\arraybackslash}X 
  | >{\centering\arraybackslash}X 
  | >{\centering\arraybackslash}X
  | >{\centering\arraybackslash}X | }
\hline
X & Y & Z & F \\
\hline
0 & 0 & 0 & 1 \\  
\hline
0 & 0 & 1 & 0 \\ 
\hline
0 & 1 & 0 & 0 \\
\hline
0 & 1 & 1 & 1 \\
\hline
1 & 0 & 0 & 1 \\  
\hline
1 & 0 & 1 & 0 \\ 
\hline
1 & 1 & 0 & 0 \\
\hline
1 & 1 & 1 & 1 \\
\hline
\end{tabularx}
\end{abstract}
\section{Components}

%\begin{table}[]
    \centering
    \begin{tabular}{ |c |c |c |c |}
\hline
\textbf{Components} & \textbf{Value} & \textbf{Quantity} \\
\hline
 Resistor & 220Ohm & 1 \\ 
 Arduino & UNO & 1 \\  
 Seven segment Display &  & 1 \\
 %Decoder& 7447&1 \\
 7447 & - & 1 \\
 Jumper wires &M-M &20\\
 Breadboard & &1\\
 \hline
 \end{tabular}
 \vspace{3mm}
 
 %\caption{Table 1.0}
    \label{table1}
%\end{table}

\section{Hardware}

%\begin{figure}
 %    \centering
 %    \includegraphics{7447_pin.png}
  %   \includegraphics{7-Segment-Display-Pinout.jpg}
%\caption{7447 pin diagram}
 %    \label{fig:7447}
  %   \textbf{Problem 2.2}  Make connections to the lower pins of
%the 7447 according to Table 2.2 and connect VCC =
%5V. 
%\hfill
%\vspace{10mm}
%\includegraphics{7-Segment-Display-Pinout.jpg}
  %   \includegraphics{7-Segment-Display-Pinout.jpg}
%\caption{seven segment diagram}
 %    \label{fig:seven}
\textbf{Problem 2.1.} Now make the connections as per
Table 2.1,2.2 and 2.3 


%\begin{table}[]
    \centering
    \begin{tabular}{ |c |c |c |c| c|}

%\textbf{Components} & \textbf{Value} & \textbf{Quantity} \\
\hline
 \textbf{} & X & Y & Z \\ 
 \textbf{Input} & 0 & 1 & 0\\
  \hline
 \end{tabular}
 Table 2.1



%\begin{table}[]
    \centering
    \begin{tabular}{ |c |c |c |c |c |c |c |c |c}

%\textbf{Components} & \textbf{Value} & \textbf{Quantity} \\
\hline
 \textbf{sevensegment} & a & b  & c & d & e & f & g \\ 
 \textbf{7447} &a'  & b' &c' &d'&e' &f' &g'\\
  \hline
 \end{tabular}
 Table 2.2
 
 
 
 %\begin{table}[]
    \centering
    \begin{tabular}{ |c |c |c |c| c|}

%\textbf{Components} & \textbf{Value} & \textbf{Quantity} \\
\hline
 \textbf{7447} & A & B & C & D \\ 
 \textbf{Arduino} &2  &-  &- &- \\
  \hline
 \end{tabular}
 \vspace{3mm}
 
 %\caption{Table 1.0}
    \label{table1}
%\end{table}
table 2.3
\section{Software}
execute the following program after
downloading.
\framebox{
\url{https://github.com/anirudhkalyan/fwc.git}}
\vspace{10mm}

X,Y,Z are the inputs that we are assigning manually in bread board and by deriving canonical form for F,
\begin{equation}
F=(X+Y+!Z)*(X+!Y+Z)*(!X+Y+!Z)*(!X+!Y+Z)
\end{equation}
%F= (X || Y ||!Z) && (X ||!Y||Z)&&(!X+Y+!Z)&&(!X||!Y||Z) 
\end{multicols}{2}
\end{document}